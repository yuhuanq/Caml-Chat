% vim:ft=tex:
%
% \documentclass[10pt]{article}
\documentclass[a4paper]{article}

\title{
    \vspace{-4.0cm}
	CS 3110: MS0: Charter
}
\author{
	Yuhuan Qiu (\texttt{yq56}),
	Eric Wang (\texttt{ew366}),
    \\Byungchan Lim (\texttt{bl458}),
	Somrita Banerjee (\texttt{sb892})
}

\begin{document}
\maketitle

\section{Logistics}

We plan on meeting regularly on Tuesdays, Thursdays, and Saturdays.
5:00-6:00 pm Tuesdays,
6:45:7:45 pm Thursdays, and
1:00-3:00 pm Saturdays.

\section{Introduction}

A customizable terminal based XMPP client/server system that provides basic
messaging, and functionality expected of an IM application, as well as additional
features such as small interactive games between users/groups.

\section{Summary}

We intend to build a instant messaging client/server application implementing
the XMPP protocol using ANSIterminal, and XML OCaml packages. There will be a
central server to which users send requests to join a certain chat room, message
a certain person, be paired with a random person or even talk to an AI/Bot. They
will then be able to send formatted text messages to the room/person. Some other
extended features are detailed in the list below. A key innovative feature of
our messenger application is that users have the ability to socialize with each
other not only through text but also by playing small games with each other like
Tic-Tac-Toe or Rock-Paper-Scissors. There will also be some customizable aspects
to our messenger client, such as keybindings.

    \subsection{Key Features}
    \begin{itemize}
        \item Talk to an AI/Bot or a random person or direct message or join a
            chat room- When a user comes online, they have four options. They
            can choose to talk to a bot that responds to messages by picking
            from a stored list of responses. They can also choose to be paired
            up with a random person who is online or directly message another
            user. They can also join a chat room from a list of available chat
            rooms that are topic-based or interest-based to have a group chat.

        \item Support sending files- Users can send files to one another.

        \item Text formatting- Users can format their
            text, add background/foreground color, underline words, etc.

        \item Read indicator- When a user has received and read a message, the
            sender will be able to see a small symbol or text message that
            indicates that their message has been read.

        \item Customizable Keybindings - To send last used message, Switch
            chats, See Favorites… Default Vi-like or Emacs like configurations.

        \item Small Games - Support small interactive games between users within
            the application, i.e. Tic-Tac-Toe, Rock-Paper-Scissors
    \end{itemize}

\end{document}

